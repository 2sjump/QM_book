\documentclass[a4j,10pt,oneside,openany]{jsbook}
%
\usepackage{amsmath,amssymb}
\usepackage{bm}
\usepackage{graphicx}
\usepackage{ascmac}
\usepackage{makeidx}
\usepackage{url}
\usepackage{braket}
\usepackage{color}
%
\makeindex
%
\newcommand{\lambdabar}{{\mkern0.75mu\mathchar '26\mkern -9.75mu\lambda}}
\newcommand{\innerprod}[2]{\bm{#1} \cdot \bm{#2}}
\newcommand{\derivative}[2]{\frac{\mathrm{d} #1}{\mathrm{d} #2}}\newcommand{\dderivative}[2]{\frac{\mathrm{d}^2 #1}{\mathrm{d} {#2}^2}}
\newcommand{\partialder}[2]{\frac{\partial #1}{\partial #2}}
%
\setlength{\textwidth}{\fullwidth}
\setlength{\textheight}{44\baselineskip}
\addtolength{\textheight}{\topskip}
\setlength{\voffset}{-0.6in}
%
\title{{\Huge \textbf{量子力学}}\\ {\small Ver. 0.0.1}}
\author{2sjump\\ \texttt{twitter : @2sjump}}
\date{\today}
\begin{document}
\maketitle
\frontmatter
\tableofcontents
\mainmatter

\setcounter{chapter}{-1} %—–0章から開始

\chapter{まえがき}
\begin{itemize}
	\item この教科書は砂川先生の教科書『量子力学』をベースに作成しています。
	\item (3元)ベクトルは太字で表します。(例:$\bm{x}, \bm{p}$)	 
	\item 説明は簡素ですので、初学には向かないかもしれません。
\end{itemize}
\include{QMtextChap01}
%\include{QMtextChap02}
\chapter{量子力学の一般原理}
\section{重ね合わせの原理: superpositon principle}
\subsection{量子力学の成立: Heisenberg の不確定性原理}
古典力学では、位置と運動量の初期条件:$\bm{x}(0), \bm{p}(0)$ を与えることで、その後の時間発展$\bm{x}(t),\ \bm{p}(t)$ を記述できることを暗に認めていた。しかし、量子力学では粒子は波動として存在するため、波動関数の広がっている空間の微小領域$\Delta x$に存在する粒子の数には$\pm 1$程度のばらつきがある。

すなわち、:
\[
	\left( \frac{\Delta k}{2\pi} \right) \Delta x \sim 1
\]
である。両辺に $\hbar$ をかけて、Einstein-de Broglie の関係:
\[
	p = \frac{\hbar}{\lambdabar}
\]
をもちいて、
\[
	\Delta p \Delta x \sim h \ \ \ : \mathrm{Heisenberg's\ uncertainty\  principle}
\]
となる。これは、位置と運動量について、同時刻に確定値を与えることを禁止するものである。

この事実は Born の確率解釈が大切なことを意味している。

今後は位置の表記に、$\bm{x}$ を使う代わりに、 $\bm{q}$ を使う。これは、表記上の都合に過ぎないが、一般に正準な変数の組として $q,\ p$ を使うことが多く、本書でもそれに倣ったためである。

\hrulefill

\subsection{離散固有値}
ある状態を表す関数 $\phi(q)$ があり、演算子$F$を作用させて新しい関数$F\phi(q)$を作ることを考える。都合のよい関数 $\phi(q)$ であれば、ある比例係数$f$をもちいて、
\[ F\phi_n(q) = f_n \phi_n(q)\]
とかけるだろう。 ここでは、そのような都合の良い関数が複数個あることを考慮して、添字$n$を付している。

このとき、関数 $\phi_n, f_n$ をそれぞれ、演算子$F$の\textbf{固有関数(eigen function), 固有値(eigenvalue(s))}とよぶ。
\\

\textcolor{blue}{
この後の部分について、以下の事項についての説明を省略します。
\begin{itemize}
	\item 演算子に対応する物理量を測定すると、その固有値はかならず実数値を返す。
	\item 重ね合わせの原理:どれかひとつの固有値$f_n$が返される。
	\item 観測によってひとつの値を得ると、状態はその固有値に対応する状態に転移している。
	\item その転移確率は展開係数$a_n$の絶対値の2乗$|a_n|^2 = a^* a$である。
	\item 完全系
	\item オブサーバブル
	\item 期待値
	\item エルミート演算子($A^\dagger = A$であるような演算子$A$をエルミート演算子と呼ぶ)
\end{itemize}
}

\hrulefill


\subsection{デルタ関数}
under construction...

\subsection{縮退のある場合}
under construction...

\hrulefill
\subsection{行列表示}
固有値方程式:\[ F\phi_n(q) = f_n \phi_n(q)\]
左から$\phi_m^*(q)$をかけて積分する。
\[
	\left[ RHS \right] \rightarrow \int \mathrm{d}q\  f_n \phi_m^* \phi_n = f_n \delta_{m,n}
\]
\[
	\left[ LHS \right] \rightarrow \int \mathrm{d}q\ \phi_m^* F \phi_n =: F_{m,n}
\]
これを行列で書けば、
\[
	\begin{pmatrix}
		F_{11} & F_{12} & \cdot \\
		F_{21} & F_{22} & \cdot \\
		\cdot & \cdot & \cdot \\
	\end{pmatrix}
	= 
	\begin{pmatrix}
		f_1 & & \\
		 & f_2 & \\
		 & & \cdot \\
	\end{pmatrix}
\]
となる。

\subsection{2状態系の固有値問題}

\subsection{エネルギー表示と位置表示}

\hrulefill 

\section{状態ベクトルとブラ・ケット記法}
前のセクションの固有値方程式:\[ F\phi_n(q) = f_n \phi_n(q)\]の固有関数を抽象的に位置$q$についてならべ、
\[
	\bm{\phi}_n = 
	\begin{pmatrix}
		\phi_n(q_1) \\
		\phi_n(q_2) \\
		\vdots \\
		\phi_n(q_\infty) \\
	\end{pmatrix}
\]
のようなベクトルを考える。厳密には位置変数は連続的な値を取るため、無限次元のこのようなベクトルで書くのは正確ではない。

上のようなベクトルを考えると、
\[
	\bm{\phi}_m^* \bm{\phi}_n = 
	\sum_i \phi_m^* (q_i) \phi_n (q_i) \rightarrow \int \mathrm{d}q\ \phi_m^*(q) \phi_n(q) = \delta_{m.n}
\]
も理解されるだろう。

ここで、$\bm{\phi}_n$は、位置$q$に無関係であった。そこで、このような関数を集めたベクトルについて、
\[\bm{\phi}_m^* \rightarrow \bra{m} \ \ \ \mathrm{bra\ vector}\]
\[\bm{\phi}_m \rightarrow \ket{n}   \ \ \ \mathrm{ket\ vector}\]
とかくことにする。

そうすれば、上の式は、
\[	\braket{m|n} = \delta_{m,n} \]
となる。
\begin{thebibliography}{20}
 \bibitem{Sunagawa} 砂川重信、量子力学、岩波書店、1991.
 \bibitem{Shimizu} 清水明、新版 量子論の基礎 その本質のやさしい理解のために、サイエンス社、2003.
\end{thebibliography}
\newpage
\printindex
%
%
\end{document}