\documentclass[a4j,10pt,oneside,openany]{jsbook}
%
\usepackage{amsmath,amssymb}
\usepackage{bm}
\usepackage{graphicx}
\usepackage{ascmac}
\usepackage{makeidx}
\usepackage{url}
\usepackage{braket}
%
\makeindex
%
\newcommand{\diff}{\mathrm{d}}  %微分記号
\newcommand{\divergence}{\mathrm{div}\,}  %ダイバージェンス
\newcommand{\grad}{\mathrm{grad}\,}  %グラディエント
\newcommand{\rot}{\mathrm{rot}\,}  %ローテーション
%
\newcommand{\lambdabar}{{\mkern0.75mu\mathchar '26\mkern -9.75mu\lambda}}
\newcommand{\innerprod}[2]{\bm{#1} \cdot \bm{#2}}
\newcommand{\partialder}[2]{\frac{\partial #1}{\partial #2}}
%
\setlength{\textwidth}{\fullwidth}
\setlength{\textheight}{44\baselineskip}
\addtolength{\textheight}{\topskip}
\setlength{\voffset}{-0.6in}
%
\title{{\Huge \textbf{量子力学}}\\ {\small Ver. 0.0.1}}
\author{2sjump\\ \texttt{tw : @2sjump}}
\date{\today}
\begin{document}
\maketitle
\frontmatter
\tableofcontents
\mainmatter

\setcounter{chapter}{-1} %—–0章から開始

\chapter{まえがき}
\begin{itemize}
	\item この教科書は砂川先生の教科書『量子力学』をベースに作成しています。
	\item (3元)ベクトルは太字で表します。(例:$\bm{x}, \bm{p}$)	 
	\item 説明は簡素ですので、初学には向かないかもしれません。
\end{itemize}

\chapter{量子力学ことはじめ}
\begin{abstract}
	この章では、量子力学が適用されるスケールについて、具体的な定数を用いて説明していきます。
\end{abstract}


\section{光の粒子性・電子の波動性}

\subsection{Compton 効果}
光の粒子性から、その粒子を  \textbf{光子, photon}  という。

角振動数$\omega$ , 波数$k$ であるとき、その電磁波は
\[
	E = \hbar \omega,\ \ p = \hbar k
\]
である photon の集団であるとみなす。

自由電子にX線を照射する。
エネルギーと運動量の保存から、
\[
	mc^2 + \hbar\omega = \sqrt{p^2c^2 + m^2c^4} + \hbar \omega',\ \ \hbar \bm{k} = \hbar \bm{k'} + \bm{k}'
\]

$\bm{p}$の消去により、

\[
	\Delta \lambdabar := \lambda - \lambda' = \lambdabar_c(1-\cos \theta)
\]

ここで \textbf{Compton 波長} :
\[
	\lambdabar_c = \frac{\hbar}{mc} \sim 10^{-13}\ \mathrm{m}
\]
を定義した。これは電子スケールの世界の基本の長さの単位となる。
\vspace{0.2in} \\*

\subsection{Bohr モデル}
量子条件:
\[
	mrv = n\hbar,\ \ \ n = 1,2,3,...
\]
振動数条件:
\[
	\hbar \omega_0 = E_{n'} - E_n
\]
電子の運動方程式(EoM: Equation of Mortion):
\[
	m \frac{v^2}{r} = \frac{e^2}{4\pi \epsilon_0^2} \frac{1}{r^2}
\]
これらの式より、
\[
	r_n = \left( \frac{\lambdabar_c}{\alpha} \right) n^2
\]
ここで、\textbf{微細構造定数, fine-structure constant} :
\[
	\alpha = \frac{e^2}{4\pi\epsilon_0 \hbar c} \sim \frac{1}{137}
\]
を定義した。微細構造定数は、電磁場にはたらく力、すなわち電磁相互作用の強さをあらわす定数である。\\
$n=1$ としたときの半径:
\[
	a_0 = \frac{\lambdabar}{\alpha} \sim 5 \times 10^{-11}\ \mathrm{m}
\]
は \textbf{Bohr 半径} と呼ばれる。

定常状態のエネルギー:
\begin{eqnarray*}
	E_n &=& \frac{1}{2}mv_n^2 - \frac{1}{4\pi\epsilon_0} \frac{e^2}{r_n} \\
	&=& \left( -\frac{mc^2}{2} \right) \frac{\alpha^2}{n^2}
\end{eqnarray*}
とくに$E_n \propto n^{-2}$ である。$n=1$をとくに、\textbf{Rydberg Energy} とよび、様々な表現がある。
\[
	E_{Ryd} := |E_{n=1}| = \left( -\frac{mc^2}{2} \right) \alpha^2
	= \frac{e^2}{8\pi\epsilon_0 a_0}
	= \left( \frac{\hbar^2}{2m} \right) \frac{1}{a_0^2}
	\sim 13.61 eV
\]

\subsection{電子の波動性}
粒子であると考えられた電子もまた、波としての性質をもつ(\textbf{電子波})。

電子の運動量の値を決める。 \textbf{Einstein-de Broglie の関係}:
\[
	p = \frac{\hbar}{\lambdabar}
\]
より、電子波の波長$\lambdabar$を測定すれば運動量がわかる。
\\

\section{Schr\"odinger 方程式(S-eq.)}
平面波:
\begin{eqnarray*}
	\psi(\bm{r},t) &=& a \exp [\ i\ (\innerprod{k}{r} - \omega t)\ ] \\ \\
	&\downarrow & \mathrm{substitute.}\ \  E=\hbar \omega, \bm{p}=\hbar\bm{k} \\ \\
	\psi(\bm{r},t) &=& a \exp [\ i\ \left( \innerprod{p'}{r}/\hbar - (\bm{p'}^2/2m) t/\hbar \right) \ ]
\end{eqnarray*}
これは以下の波動方程式の解である。
\[
	i\hbar \partialder{\psi}{t} = \frac{-\hbar^2}{2m} \nabla^2 \psi
\]
ここで、古典論に対して以下の置き換えをする。
\[
	E \rightarrow i\hbar \partialder{}{t}, \ \ p_w \rightarrow \frac{\hbar}{i} \partialder{}{w}\ \ (w = x, y, z)
\]
先の波動方程式は、
\[
	i\hbar \partialder{\psi}{t} = H\psi
\]
となる。ここで\textbf{Hamiltonian}とよばれる演算子は、
\[
	H = \frac{-\hbar^2}{2m} \nabla^2 +V(\bm{r})
\]
と与えられる。
\begin{thebibliography}{20}
 \bibitem{Sunagawa} 砂川重信、量子力学、岩波書店、1991.
 \bibitem{Shimizu} 清水明、新版 量子論の基礎 その本質のやさしい理解のために、サイエンス社、2003
\end{thebibliography}
\newpage
\printindex
%
%
\end{document}